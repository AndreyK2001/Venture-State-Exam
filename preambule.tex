\usepackage{luatex85}
\pdfminorversion=5

\usepackage{fontspec}
\usepackage{polyglossia}
\usepackage[a4paper, 
lmargin=25mm, rmargin=15mm, tmargin=20mm, bmargin=20mm]{geometry}
\usepackage{multirow}

\usepackage{pgfplots}
\pgfplotsset{compat=1.18}

\setdefaultlanguage{russian}
\setotherlanguage{english}

% Согласно требованиям к ВКР
\defaultfontfeatures{Ligatures=TeX}

\setmainfont{Times New Roman}
\setmonofont{Courier New}
\setsansfont{Arial}

\newfontfamily\cyrillicfont{Times New Roman}
\newfontfamily\cyrillicfontsf{Arial}
\newfontfamily\cyrillicfonttt{Courier New}

\newfontfamily\englishfont{Times New Roman}
\newfontfamily\englishfontsf{Arial}
\newfontfamily\englishfonttt{Courier New}

\linespread{1.5}

\usepackage{titlesec}

\titleformat{\section}{\normalfont\fontsize{14}{14}\bfseries}{\thesection}{1em}{}
\titleformat{\subsection}{\normalfont\fontsize{14}{14}\bfseries}{\thesubsection}{1em}{}
\titleformat{\subsubsection}{\normalfont\fontsize{14}{14}\bfseries}{\thesubsubsection}{1em}{}

\setlength{\parindent}{1.25cm}

\renewcommand\thesection{\arabic{section}}

\usepackage[a-1b,mathxmp]{pdfx}
\hypersetup{
    hidelinks,
    colorlinks=true,
    linkcolor=black,    % Цвет гиперссылок
    citecolor=black,    % Цвет ссылок на цитаты
    urlcolor=black      % Цвет ссылок на URL
}


\usepackage[backend=biber,
bibencoding=utf8,
sorting=none,
style=gost-numeric,
language=autobib,
autolang=other,
clearlang=true,
defernumbers=true,
sortcites=true,
doi=true,
isbn=true,
]{biblatex}

\renewcommand{\UrlFont}{\small\rmfamily\tt}
\appto{\bibsetup}{\raggedright}


% \bibliography{bibliography/betti}
%\bibliography{bibliography/}


\usepackage{amsfonts}

\usepackage{tikz}
\usepackage{pgfplots}
\usepackage{wrapfig}

\usepackage{caption}
\usepackage{subcaption}

\usepackage{amsfonts, amssymb, amsmath, amsthm}
\usepackage{mathtools}
\usepackage{enumerate}
\usepackage{verbatim}

\usepackage{mathrsfs,amsfonts,mathtools}
\usepackage{amsmath}          
\usepackage{amssymb}

\usepackage{amsthm} 
\usepackage{thmtools}

\usepackage{graphicx} 
\graphicspath{{images/}}

\renewcommand{\emptyset}{\varnothing}
\renewcommand{\epsilon}{\varepsilon}
\renewcommand{\phi}{\varphi}
\renewcommand{\kappa}{\varkappa}


\hyphenpenalty=10000 % Запрет переноса
\exhyphenpenalty=10000 % Запрет переноса после дефиса
\brokenpenalty=10000 % Запрет переноса на конце страницы

\pretolerance=500
\tolerance=1500
\emergencystretch=15pt
\sloppy


