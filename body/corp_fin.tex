\section{Корпоративные финансы}

\subsection{Основы финансовой отчетности: баланс, ОФР, ОДДС, амортизационная политика компании и её связь с денежным потоком}

\textbf{Баланс}
Баланс --- это отчет, отражающий финансовое положение компании на определенную дату. Он состоит из двух частей:
\begin{itemize}
    \item \textbf{Активы} --- ресурсы, контролируемые компанией (основные средства, оборотные средства и т.д.).
    \item \textbf{Пассивы и капитал} --- источники финансирования (обязательства и собственный капитал).
\end{itemize}
Основное равенство баланса: 
\begin{equation}
    \text{Активы} = \text{Обязательства} + \text{Собственный капитал}.
\end{equation}

\textbf{Отчет о финансовых результатах (ОФР)}

ОФР показывает доходы, расходы и чистую прибыль компании за определенный период. Основные показатели:
\begin{itemize}
    \item Выручка.
    \item Себестоимость.
    \item Операционная прибыль.
    \item Чистая прибыль.
\end{itemize}

Формула расчета чистой прибыли:
\begin{equation}
    \text{Чистая прибыль} = \text{Выручка} - \text{Себестоимость} - \text{Операционные расходы} - \text{Проценты и налоги}.
\end{equation}


\textbf{Отчет о движении денежных средств (ОДДС)}

ОДДС отражает движение денежных средств за период, разделенное на три категории:
\begin{itemize}
    \item Операционная деятельность --- денежные потоки от основной деятельности (выручка, расходы).
    \item Инвестиционная деятельность --- денежные потоки от приобретения и выбытия активов.
    \item Финансовая деятельность --- потоки, связанные с займами и собственным капиталом.
\end{itemize}

При составлении CFS используются прямой или косвенный методы.

\begin{itemize}
    \item \textbf{Прямой метод:} последовательно показываются все основные виды валовых денежных поступлений и валовых денежных платежей, разница которых составляет чистый приток или отток денежных средств
    \item \textbf{Косвенный метод:} корректировка чистой прибыли или убытка с учетом результатов операций неденежного характера, любых отсрочек или начислений прошлых периодов, будущих денежных поступлений или платежей.
\end{itemize}

Различия в формировании CF прямым и косвенным методом затрагивают только раздел расчета денежного потока по операционной деятельности. Расчет денежных потоков по инвестиционной и финансовой деятельности одинаковы в обоих случаях.

ОДДС помогает понять, как компания генерирует денежные средства и управляет ликвидностью.

\textbf{Амортизационная политика компании и её влияние на денежный поток}

Амортизационная политика определяет методы списания стоимости основных средств на расходы:
\begin{itemize}
    \item \textbf{Линейный метод} --- равномерное списание стоимости.
    \item \textbf{Ускоренная амортизация} --- списание большего объема расходов в начале срока эксплуатации.
\end{itemize}
Амортизация не является денежным расходом, но уменьшает налогооблагаемую базу, что увеличивает чистый денежный поток:
\begin{equation}
    \text{Чистый денежный поток} = \text{Чистая прибыль} + \text{Амортизация} - \text{Капитальные затраты}.
\end{equation}

\textbf{Взаимосвязь 3 видов отчетности}

\begin{enumerate}
    \item Чистая прибыль из ОФР отражается в разделе "Собственный капитал" баланса. Нераспределенная прибыль увеличивается на сумму чистой прибыли за отчетный период. В случае убытка уменьшается.
    \item Денежные потоки из ОДДС объясняют изменения в денежных средствах баланса.
    \item Амортизация влияет на операционные расходы ОФР и, соответственно, на операционные денежные потоки в ОДДС. Она увеличивает операционные денежные потоки.
\end{enumerate}



\pagebreak
\subsection{Долевые ценные бумаги. Акции. Акции простые и привилегированные. Дивиденды. Номинальная и курсовая стоимость акций. Модель дисконтирования дивидендов для оценки акций}

\textbf{Акции}

Акция --- это долевая ценная бумага, подтверждающая право её владельца на:
\begin{itemize}
    \item Участие в управлении компанией.
    \item Получение дивидендов.
    \item Долю имущества при ликвидации компании.
\end{itemize}

\textit{Простые акции}

Простые акции предоставляют владельцам право голоса на общем собрании акционеров. Однако выплаты дивидендов не гарантированы и зависят от решения компании.

\textit{Привилегированные акции}

Привилегированные акции не дают право голоса, но обеспечивают фиксированный размер дивидендов. В случае ликвидации компании их владельцы имеют приоритет перед держателями простых акций в распределении имущества.

\textbf{Дивиденды}

Дивиденды --- это часть прибыли компании, распределяемая между акционерами. Основные виды:
\begin{itemize}
    \item \textbf{Обычные дивиденды} --- регулярные выплаты из прибыли.
    \item \textbf{Специальные дивиденды} --- разовые выплаты.
    \item \textbf{Кумулятивные дивиденды} --- применимы для привилегированных акций и накапливаются в случае пропуска выплат.
\end{itemize}

\textbf{Номинальная и курсовая стоимость акций}
\begin{itemize}
    \item \textbf{Номинальная стоимость} --- фиксированная стоимость, указанная в уставе компании. Используется для определения уставного капитала.
    \item \textbf{Курсовая стоимость} --- рыночная цена акции, определяемая спросом и предложением на рынке.
\end{itemize}

\textbf{Модель дисконтирования дивидендов (DDM)}

Модель дисконтирования дивидендов используется для оценки справедливой стоимости акции. Основная формула:
\begin{equation}
    P_0 = \sum_{t=1}^{\infty} \frac{D_t}{(1 + r)^t},
\end{equation}
\noindent где:
\begin{itemize}
    \item $P_0$ --- текущая стоимость акции,
    \item $D_t$ --- дивиденды в году $t$,
    \item $r$ --- ставка дисконтирования.
\end{itemize}

Для акций с постоянным ростом дивидендов используется упрощённая формула Гордона:

\begin{equation}
    P_0 = \sum_{t=1}^{\infty} D_1 \frac{(1 + g)^t}{(1 + r)^t} = D_1 \frac{1+g}{r-g},
\end{equation}

\noindent где $g$ --- темп роста дивидендов.


\pagebreak
\subsection{Долговые ценные бумаги. Облигации. Купонные и бескупонные облигации. Ставка дисконтирования. Доходность облигаций. Экс-дивидендный период для акций и облигаций}

\textbf{Облигации}

Облигация --- это долговая ценная бумага, удостоверяющая обязательство эмитента вернуть номинальную стоимость и уплатить проценты (купон) в соответствии с условиями выпуска.

\textit{Купонные облигации}

Купонные облигации предусматривают регулярные процентные выплаты (купон) в течение срока обращения.

\textit{Бескупонные облигации}

Бескупонные облигации не предполагают регулярных выплат. Они продаются со скидкой к номинальной стоимости, а доход инвестора формируется за счёт разницы между ценой покупки и номиналом, выплачиваемым при погашении.

\textbf{Ставка дисконтирования}

Ставка дисконтирования --- это ставка, используемая для приведения будущих денежных потоков к текущей стоимости. Она определяется уровнем риска и условиями рынка. Формула для оценки текущей стоимости облигации:
\begin{equation}
    P = \sum_{t=1}^{n} \frac{C_t}{(1 + r)^t} + \frac{N}{(1 + r)^n},
\end{equation}
\noindent где:
\begin{itemize}
    \item $P$ --- текущая стоимость облигации,
    \item $C_t$ --- купонные выплаты в году $t$,
    \item $N$ --- номинальная стоимость,
    \item $r$ --- ставка дисконтирования,
    \item $n$ --- срок до погашения.
\end{itemize}

\textbf{Доходность облигаций}

Доходность облигаций характеризует уровень дохода, который инвестор получит от владения облигацией. Основные виды доходности:
\begin{itemize}
    \item \textbf{Текущая доходность}:
    \begin{equation}
        \text{Текущая доходность} = \frac{\text{Купонный доход}}{\text{Цена покупки}} \times 100\%.
    \end{equation}
    \item \textbf{Доходность по бескупонным облигациям:}
    \begin{equation}
        r = \left( \sqrt[n]{\frac{N}{P}} - 1 \right) * 100 \%
    \end{equation}
    \item \textbf{Доходность к погашению (YTM)} --- отражает внутреннюю норму доходности, учитывая все будущие платежи и номинальную стоимость.
    \begin{equation}
        D = \frac{\sum \limits ^T _{t=1} PV(C_t)}{P_0},
    \end{equation}
    \noindent где $PV(C_t)$ --- приведённая стоимость платажей, которые будут получены в момент времени $t$, $P_0$ --- номинал облигации,  $T$ -- срок погашения.
\end{itemize}

\textbf{Экс-дивидендный период}
Экс-дивидендный период --- это период времени, в течение которого покупатель ценных бумаг уже не имеет права на получение ближайших выплат:
\begin{itemize}
    \item Для \textbf{акций} --- право на дивиденды принадлежит владельцу на дату фиксации реестра, даже если акция была продана после этой даты.
    \item Для \textbf{облигаций} --- применяется аналогичный принцип для купонных выплат.
\end{itemize}

\pagebreak
\subsection{Кредиты. Ковенанты. Субординация займов}

\textbf{Кредиты}
Кредит --- это форма заемного капитала, предоставляемого кредитором заемщику на определенных условиях возврата и оплаты процентов. Основные характеристики кредита:
\begin{itemize}
    \item \textit{Срок кредита}: краткосрочные, среднесрочные и долгосрочные.
    \item \textit{Процентная ставка}: фиксированная или плавающая.
    \item \textit{График погашения}: аннуитетные или дифференцированные платежи.
    \item \textit{Обеспечение}: залог, поручительство или необеспеченные кредиты.
\end{itemize}

\textbf{Ковенанты}
Ковенанты --- это обязательства или ограничения, установленные кредитным договором, которые заемщик должен соблюдать в течение срока действия кредита. Разделяются на:
\begin{itemize}
    \item \textit{Позитивные ковенанты}: требования, которые заемщик обязан выполнять (например, предоставление отчетности).
    \item \textit{Негативные ковенанты}: ограничения на действия заемщика (например, запрет на дополнительное привлечение долгов).
    \item \textit{Финансовые ковенанты}: поддержание определённых финансовых показателей, таких как коэффициент покрытия долга ($\text{Debt/EBITDA}$) или уровень ликвидности.
\end{itemize}
Нарушение ковенант может привести к досрочному погашению кредита или пересмотру его условий.

\textbf{Субординация займов}
Субординация займов --- это порядок очередности удовлетворения требований кредиторов в случае ликвидации компании. Основные уровни:
\begin{itemize}
    \item \textit{Старший долг (Senior debt)} --- имеет приоритет перед другими обязательствами.
    \item \textit{Субординированный долг (Subordinated debt)} --- удовлетворяется после старшего долга.
    \item \textit{Между собой субординированные займы} могут делиться на категории по приоритету.
\end{itemize}
Субординация повышает уровень риска для кредиторов с более низким приоритетом, но зачастую компенсируется более высокой доходностью.


\pagebreak
\subsection{Финансовый анализ: показатели ЧОК и EBITDA, анализ ликвидности, финансовой устойчивости и финансовой эффективности}

\textbf{Показатели ЧОК и EBITDA}

\textit{Чистый Операционный Кэш (ЧОК)}

ЧОК (\textit{Net Operating Cash Flow}) показывает денежные средства, генерируемые операционной деятельностью компании. Формула:
\begin{equation}
    \text{ЧОК} = \text{Чистая прибыль} + \text{Амортизация} + \text{Изменение оборотного капитала}.
\end{equation}
ЧОК отражает способность компании генерировать денежные средства для обслуживания долга и инвестиций.

\textit{EBITDA}

EBITDA (\textit{Earnings Before Interest, Taxes, Depreciation, and Amortization}) --- показатель операционной прибыли до вычета процентов, налогов и амортизации. Формула:
\begin{equation}
    \text{EBITDA} = \text{Выручка} - \text{Операционные расходы}.
\end{equation}
EBITDA используется для оценки операционной эффективности компании независимо от её структуры финансирования.

\textbf{Анализ ликвидности}

Ликвидность характеризует способность компании своевременно погашать краткосрочные обязательства. Основные коэффициенты:
\begin{itemize}
    \item \textbf{Коэффициент текущей ликвидности}:
    \begin{equation}
        \text{Текущая ликвидность} = \frac{\text{Оборотные активы}}{\text{Краткосрочные обязательства}}.
    \end{equation}
    Оптимальное значение: $1.5$–$2.0$.
    \item \textbf{Коэффициент быстрой ликвидности}:
    \begin{equation}
        \text{Быстрая ликвидность} = \frac{\text{Оборотные активы} - \text{Запасы}}{\text{Краткосрочные обязательства}}.
    \end{equation}
    Оптимальное значение: $1.0$–$1.5$.
\end{itemize}

\textbf{Анализ финансовой устойчивости}

Финансовая устойчивость отражает соотношение собственных и заемных средств компании. Основные показатели:
\begin{itemize}
    \item \textbf{Коэффициент автономии}:
    \begin{equation}
        \text{Автономия} = \frac{\text{Собственный капитал}}{\text{Общие активы}}.
    \end{equation}
    Оптимальное значение: $\geq 0.5$.
    \item \textbf{Коэффициент задолженности}:
    \begin{equation}
        \text{Задолженность} = \frac{\text{Заемный капитал}}{\text{Собственный капитал}}.
    \end{equation}
    Оптимальное значение: $\leq 1.0$.
\end{itemize}

\textbf{Анализ финансовой эффективности}

Финансовая эффективность оценивается через рентабельность. Основные коэффициенты:
\begin{itemize}
    \item \textbf{Рентабельность продаж}:
    \begin{equation}
        \text{Рентабельность продаж} = \frac{\text{Чистая прибыль}}{\text{Выручка}} \times 100\%.
    \end{equation}
    \item \textbf{Рентабельность активов (ROA)}:
    \begin{equation}
        \text{ROA} = \frac{\text{Чистая прибыль}}{\text{Общие активы}} \times 100\%.
    \end{equation}
    \item \textbf{Рентабельность собственного капитала (ROE)}:
    \begin{equation}
        \text{ROE} = \frac{\text{Чистая прибыль}}{\text{Собственный капитал}} \times 100\%.
    \end{equation}
\end{itemize}

\pagebreak
\subsection{Структура капитала: теории структуры капитала, средневзвешанная стоимость капитала (WACC), модель ценообразования активов (CAPM)}

\textit{Теории структуры капитала}
\textbf{Теорема Модильяни-Миллера (MM)}

Основные положения:
\begin{itemize}
    \item При отсутствии налогов и рыночных фрикций структура капитала не влияет на стоимость компании.
    \item С учетом налогов стоимость компании увеличивается при росте доли заемного капитала из-за налогового щита.
\end{itemize}

\textbf{Теория компромисса}

Компания стремится к оптимальной структуре капитала, балансируя между выгодами налогового щита и затратами финансового риска (например, риска банкротства).

\textbf{Теория иерархии финансирования}

Компании предпочитают финансироваться:
\begin{enumerate}
    \item За счет нераспределенной прибыли.
    \item Заемным капиталом.
    \item Выпуском новых акций.
\end{enumerate}

\textbf{Средневзвешенная стоимость капитала (WACC)}

WACC (\textit{Weighted Average Cost of Capital}) --- это средняя стоимость привлеченных средств, взвешенная по их доле в капитале. Формула:
\begin{equation}
    \text{WACC} = \frac{E}{E + D} \cdot r_e + \frac{D}{E + D} \cdot r_d \cdot (1 - T),
\end{equation}
\noindent где:
\begin{itemize}
    \item $E$ --- собственный капитал,
    \item $D$ --- заемный капитал,
    \item $r_e$ --- стоимость собственного капитала,
    \item $r_d$ --- стоимость заемного капитала,
    \item $T$ --- ставка налога.
\end{itemize}

WACC используется для оценки стоимости компании и анализа инвестиционных проектов.

\textbf{Модель ценообразования активов (CAPM)}

CAPM (\textit{Capital Asset Pricing Model}) определяет ожидаемую доходность актива на основе его систематического риска. Формула:
\begin{equation}
    r_e = r_f + \beta \cdot (r_m - r_f),
\end{equation}
\noindent где:
\begin{itemize}
    \item $r_e$ --- ожидаемая доходность актива,
    \item $r_f$ --- безрисковая ставка,
    \item $\beta$ --- коэффициент бета (показатель рыночного риска актива),
    \item $r_m$ --- рыночная доходность.
\end{itemize}

CAPM помогает определить стоимость собственного капитала, необходимую для расчета WACC.
