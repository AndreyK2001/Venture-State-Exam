\section{Управление интеллектуальной собственностью}

\subsection{Основные объекты права интеллектуальной собственности (результаты интеллектуальной деятельности, далее - РИД). Специфика правовой охраны объектов патентного права, объектов авторского права (прав смежных с авторскими) и средств индивидуализации юридических лиц, товаров, работ, услуг и предприятий}

\textbf{Основные объекты права интеллектуальной собственности}

\textit{Результаты интеллектуальной деятельности (РИД)}

\begin{itemize}
    \item \textbf{Объекты патентного права:}
    \begin{itemize}
        \item Изобретения;
        \item Полезные модели;
        \item Промышленные образцы.
    \end{itemize}
    \item \textbf{Объекты авторского права:}
    \begin{itemize}
        \item Литературные, музыкальные, художественные произведения;
        \item Программы для ЭВМ;
        \item Базы данных.
    \end{itemize}
    \item \textbf{Объекты смежных прав:}
    \begin{itemize}
        \item Исполнения;
        \item Фонограммы;
        \item Передачи организаций эфирного или кабельного вещания.
    \end{itemize}
\end{itemize}

\textit{Средства индивидуализации}

\begin{itemize}
    \item Товарные знаки (знаки обслуживания);
    \item Наименования мест происхождения товаров;
    \item Фирменные наименования;
    \item Коммерческие обозначения.
\end{itemize}

\textbf{Специфика правовой охраны}

\textit{Объекты патентного права}

\begin{itemize}
    \item Предоставляется охрана на основе патента;
    \item Условия патентоспособности:
    \begin{itemize}
        \item Новизна;
        \item Изобретательский уровень;
        \item Промышленная применимость.
    \end{itemize}
    \item Срок охраны:
    \begin{itemize}
        \item Изобретения --- 20 лет;
        \item Полезные модели --- 10 лет;
        \item Промышленные образцы --- 5 лет (с возможностью продления).
    \end{itemize}
\end{itemize}

\textit{Объекты авторского права и смежных прав}

\begin{itemize}
    \item Автоматическая охрана с момента создания произведения;
    \item Срок действия авторского права:
    \begin{itemize}
        \item В течение жизни автора и 70 лет после его смерти;
    \end{itemize}
    \item Смежные права:
    \begin{itemize}
        \item Действуют 50 лет с момента исполнения или записи.
    \end{itemize}
\end{itemize}

\textit{Средства индивидуализации}
\begin{itemize}
    \item Регистрация обязательна (кроме коммерческих обозначений);
    \item Срок охраны:
    \begin{itemize}
        \item Товарные знаки --- 10 лет (с возможностью продления);
        \item Наименования мест происхождения --- бессрочная охрана при поддержании прав.
    \end{itemize}
\end{itemize}

\pagebreak
\subsection{Основные способы распоряжения интеллектуальными правами на РИД. Понятия: исключительное право (имущественное право), личные неимущественные права и иные права. Служебные РИД (специфика правовой охраны)}

\textbf{Основные понятия}

\textit{Исключительное право (имущественное право)}
\begin{itemize}
    \item Даёт правообладателю возможность использовать объект ИС любым не противоречащим закону способом.
    \item Возможность распоряжения:
    \begin{itemize}
        \item Передача исключительного права (отчуждение);
        \item Предоставление права использования (лицензия).
    \end{itemize}
    \item Срок действия зависит от объекта:
    \begin{itemize}
        \item Авторские права --- 70 лет после смерти автора;
        \item Патенты --- от 5 до 20 лет в зависимости от вида.
    \end{itemize}
\end{itemize}

\textit{Личные неимущественные права}
\begin{itemize}
    \item Связаны с личностью автора:
    \begin{itemize}
        \item Право авторства;
        \item Право на имя;
        \item Право на неприкосновенность произведения.
    \end{itemize}
    \item Неотчуждаемы и действуют бессрочно.
\end{itemize}

\textit{Иные права}
\begin{itemize}
    \item Право следования (для произведений искусства);
    \item Право на обнародование;
    \item Право на отзыв (в авторском праве).
\end{itemize}

\textbf{Основные способы распоряжения интеллектуальными правами}

\textit{Отчуждение исключительного права}
\begin{itemize}
    \item Полная передача исключительных прав другому лицу;
    \item Оформляется договором об отчуждении.
\end{itemize}

\textit{Лицензионные договоры}
\begin{itemize}
    \item \textbf{Виды лицензий:}
    \begin{itemize}
        \item Исключительная лицензия --- предоставляет исключительное право использования;
        \item Простая (неисключительная) лицензия --- позволяет использовать объект ИС наряду с другими лицами.
    \end{itemize}
    \item Устанавливаются сроки, способы использования, территория действия.
\end{itemize}

\textit{Договоры коммерческой концессии (франчайзинг)}
\begin{itemize}
    \item Использование комплекса прав, включающего товарные знаки, коммерческую информацию и технологии.
\end{itemize}

\textit{Наследование}
\begin{itemize}
    \item Исключительное право может быть передано наследникам.
\end{itemize}

\textbf{Служебные РИД}

Служебные РИД --- РИД, созданные работником в рамках трудовых обязанностей или по заданию работодателя.

\textit{Специфика правовой охраны}

\begin{itemize}
    \item Исключительное право принадлежит работодателю, если иное не предусмотрено договором;
    \item Автор сохраняет личные неимущественные права;
    \item Работнику выплачивается вознаграждение за использование РИД.
\end{itemize}

\pagebreak
\subsection{Основные категории нематериальных активов: патенты, лицензии, товарные знаки и др}

\begin{enumerate}
    \item Патенты
        \begin{itemize}
            \item Предоставляют исключительное право на использование изобретений, полезных моделей и промышленных образцов.
            \item Условия охраны:
            \begin{itemize}
                \item Новизна;
                \item Изобретательский уровень;
                \item Промышленная применимость.
            \end{itemize}
            \item Срок действия:
            \begin{itemize}
                \item Изобретения — 20 лет;
                \item Полезные модели — 10 лет;
                \item Промышленные образцы — 5 лет (с возможностью продления).
            \end{itemize}
        \end{itemize}
    \item Лицензии
        \begin{itemize}
            \item Права на использование объектов ИС, предоставляемые по лицензионным договорам.
            \item \textbf{Виды лицензий:}
            \begin{itemize}
                \item Исключительная лицензия;
                \item Простая (неисключительная) лицензия.
            \end{itemize}
        \end{itemize}
    \item Товарные знаки
        \begin{itemize}
            \item Обозначения, служащие для индивидуализации товаров, работ или услуг.
            \item Правовая охрана:
            \begin{itemize}
                \item Регистрация в национальном или международном реестре;
                \item Срок действия — 10 лет с возможностью продления.
            \end{itemize}
        \end{itemize}
    \item Авторские права и смежные права
        \begin{itemize}
            \item Объекты:
            \begin{itemize}
                \item Литературные и художественные произведения;
                \item Программы для ЭВМ и базы данных;
                \item Исполнения, фонограммы, эфирные передачи.
            \end{itemize}
            \item Срок действия:
            \begin{itemize}
                \item Авторские права — 70 лет после смерти автора;
                \item Смежные права — 50 лет.
            \end{itemize}
        \end{itemize}
    \item Коммерческие тайны (ноу-хау)
        \begin{itemize}
            \item Информация, имеющая коммерческую ценность и защищённая от несанкционированного использования.
            \item Условия охраны:
            \begin{itemize}
                \item Конфиденциальность информации;
                \item Экономическая ценность.
            \end{itemize}
        \end{itemize}
    \item Прочие нематериальные активы
        \begin{itemize}
            \item Наименования мест происхождения товаров;
            \item Фирменные наименования;
            \item Деловая репутация (гудвилл).
        \end{itemize}
\end{enumerate}

\pagebreak
\subsection{Способы управления правами на РИД в организации}

\begin{enumerate}
    \item Идентификация и учёт РИД
        \begin{itemize}
            \item Выявление созданных в организации объектов интеллектуальной собственности (ИС).
            \item Постановка РИД на учёт как нематериальных активов.
            \item Регистрация прав на РИД в государственных реестрах (патенты, товарные знаки и др.).
        \end{itemize}
    \item Правовая защита РИД
        \begin{itemize}
            \item Оформление прав:
            \begin{itemize}
                \item Патентование изобретений, полезных моделей, промышленных образцов;
                \item Регистрация товарных знаков и наименований мест происхождения товаров;
                \item Авторское право на программы для ЭВМ, базы данных и произведения.
            \end{itemize}
            \item Поддержание прав:
            \begin{itemize}
                \item Своевременная уплата пошлин;
                \item Пролонгация охраны (например, для товарных знаков).
            \end{itemize}
        \end{itemize}
    \item Договорное управление
        \begin{itemize}
            \item Лицензионные договоры:
            \begin{itemize}
                \item Передача прав на использование РИД (исключительные и неисключительные лицензии).
            \end{itemize}
            \item Договоры об отчуждении:
            \begin{itemize}
                \item Полная передача исключительных прав другому лицу.
            \end{itemize}
            \item Франчайзинг:
            \begin{itemize}
                \item Передача комплекса прав, включая товарные знаки и технологии.
            \end{itemize}
        \end{itemize}
    \item  Внутренние меры по управлению РИД
        \begin{itemize}
            \item Создание и внедрение политики управления ИС:
            \begin{itemize}
                \item Установление правил разработки служебных РИД;
                \item Закрепление прав на служебные РИД за работодателем.
            \end{itemize}
            \item Введение стандартов по охране коммерческой тайны и ноу-хау:
            \begin{itemize}
                \item Доступ к конфиденциальной информации только уполномоченным лицам;
                \item Составление соглашений о неразглашении (NDA).
            \end{itemize}
        \end{itemize}
    \item Монетизация РИД
        \begin{itemize}
            \item Лицензионные платежи и роялти.
            \item Продажа исключительных прав на РИД.
            \item Использование РИД в качестве вклада в уставный капитал.
        \end{itemize}
    \item Управление рисками
        \begin{itemize}
            \item Мониторинг нарушений прав на РИД:
            \begin{itemize}
                \item Судебная защита прав;
                \item Претензионная работа с нарушителями.
            \end{itemize}
            \item Анализ рынка на наличие конфликтующих объектов ИС.
        \end{itemize}
    
\end{enumerate}






