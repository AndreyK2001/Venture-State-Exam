\section{Микроэкономика и инвестиционная оценка}

\subsection{Теория массового обслуживания. Стандартная модель M/M/1, M/M/n. Детерминированное время обслуживания M/D/1 и другие распределения. Роль цепей поставок в обеспечении конкурентного преимущества}



\subsection{Классическая структура цепи поставок, принципы её организации. Принципы управления запасами в цепи поставок (детерминированный и стохастический спрос). Различные типы договоров поставки для оптимизации работы всей цепи поставок}


\subsection{Финансовая модель технологического проекта. Цели финансового моделирования. Основные принципы и подходы к построению финансовой модели. Основные формы финансовой отчетности, их цели, задачи и структура. Взаимосвязь форм финансовой отчетности. Особенности отражения в отчетности расходов на R\&D и создания НМА.}

\textbf{Финансовая модель технологического проекта} --- это систематизированная модель, отражающая основные параметры и логику будущей деятельности компании в технологической сфере. В ней учитываются ожидания по выручке и рынку, особенности затрат, структура финансирования и перспективы роста стоимости бизнеса.

\textbf{Цели финансового моделирования:}
\begin{enumerate}
    \item \textit{Прогнозирование:} дать количественную оценку развития проекта (продажи, расходы, прибыль).
    \item \textit{Привлечение инвестиций:} обосновать потребность в финансировании и продемонстрировать инвестиционную привлекательность.
    \item \textit{Планирование и бюджетирование:} определить ключевые ресурсы и распределить их по этапам развития.
    \item \textit{Управление рисками:} выявить наиболее уязвимые места и проверить проект на устойчивость к неблагоприятным факторам.
\end{enumerate}

\textbf{Основные принципы и подходы к построению финансовой модели:}
\begin{enumerate}
    \item \textit{Прозрачность и детальность:} модель должна содержать понятные ссылки на исходные данные (допущения, прогнозы спроса, ставок и т.д.).
    \item \textit{Гибкость:} возможность быстро менять параметры (стоимость оборудования, темпы роста рынка), чтобы проводить сценарный и чувствительный анализ.
    \item \textit{Реалистичность:} использование правдоподобных предположений с учётом отраслевой и макроэкономической специфики.
    \item \textit{Взаимосвязанность блоков:} операционный блок (выручка, расходы), инвестиционный блок (капитальные затраты, R\&D), финансирование (кредиты, долевое участие), оценка окупаемости и распределение денежных потоков.
\end{enumerate}

\textbf{Основные формы финансовой отчетности:}
\begin{enumerate}
    \item \textit{Отчет о прибыли и убытках (P\&L)}
    
    Цель: показать доходы и расходы за период, определить финансовый результат (прибыль или убыток).
    
    Структура: выручка $\rightarrow$ себестоимость $\rightarrow$ валовая прибыль $\rightarrow$ операционные и прочие расходы $\rightarrow$ налоги $\rightarrow$ чистая прибыль.

    \item \textit{Баланс (Balance Sheet)}

    Цель: отразить финансовое состояние компании на определённую дату.
    
    Структура: активы (оборотные и внеоборотные), обязательства (краткосрочные и долгосрочные), собственный капитал.

    \item \textit{Отчет о движении денежных средств (Cash Flow Statement)}
    
    Цель: показать реальные потоки денежных средств за период, их источники и использование.
    
    Структура: операционная деятельность (CFO), инвестиционная деятельность (CFI), финансовая деятельность (CFF).
    
\end{enumerate}

\textbf{Взаимосвязь форм финансовой отчетности}

\begin{enumerate}
    \item \textit{Чистая прибыль} из P\&L переносится в раздел собственного капитала Баланса и влияет на нераспределенную прибыль.
    \item \textit{Движение денежных средств} зависит от чистой прибыли (корректировка на неденежные статьи) и изменений в статьях Баланса (например, дебиторская задолженность, запасы, кредиторская задолженность).
    \item \textit{Инвестиции в активы} отражаются в Балансе (например, увеличение нематериальных активов) и в разделе инвестиционной деятельности Отчета о движении денежных средств.
\end{enumerate}


\textbf{Особенности отражения в отчетности расходов на R\&D и создания НМА}
\textit{Расходы на R\&D} часто включают заработную плату, материалы, лабораторные расходы. В зависимости от стандартов (МСФО/РСБУ/GAAP) часть затрат может капитализироваться в нематериальные активы (НМА), если ожидается будущая экономическая выгода и соблюдены критерии признания актива.
\textit{Капитализированная часть R\&D} становится нематериальным активом в Балансе и амортизируется в течение срока полезного использования.
\textit{Некапитализированные затраты} отражаются как расходы периода в P\&L, что снижает чистую прибыль.









\subsection{Оценка инвестиционной привлекательности и эффективности технологического проекта. Расчет и экономический смысл коэффициентов эффективности проекта: NPV, IRR (MIRR), период окупаемости и дисконтированный период окупаемости, индекс прибыльности}

\subsection{Основные методы определения стоимости бизнеса. Понятие стоимости pre- и post-money. Принцип дисконтирования денежных потоков: понятие стоимости денег во времени, текущая и будущая стоимость. Определение ставки дисконтирования. Прогнозная и постпрогнозная (терминальная) стоимость. Модель Гордона для определения терминальной стоимости. Применимость доходного подхода для молодых инновационных компаний}

\subsection{Оценка бизнеса методом венчурного капитала (в т. ч. в случае нескольких раундов инвестирования). Сравнительный подход (мультипликаторы): методы и порядок применения. Применимость сравнительного подхода для молодых инновационных компаний}
