\section{Микроэкономика и инвестиционная оценка}

\subsection{Теория массового обслуживания. Стандартная модель M/M/1, M/M/n. Детерминированное время обслуживания M/D/1 и другие распределения. Роль цепей поставок в обеспечении конкурентного преимущества}

Теория массового обслуживания (или теория очередей) изучает потоки заявок и процесс их обслуживания ограниченными ресурсами. Ключевые параметры в таких моделях:
\begin{enumerate}
    \item Интенсивность входящего потока $\lambda$ --- среднее число заявок в единицу времени.
    \item Интенсивность обслуживания $\mu$ --- среднее число заявок, обслуживаемых системой в единицу времени.
    \item Количество каналов обслуживания (серверов) --- может быть один или несколько.
\end{enumerate}

Главная цель --- найти вероятностные характеристики очереди (например, среднюю длину очереди, среднее время ожидания) и оценить пропускную способность и степень загрузки системы.

\textbf{Модель M/M/1}

\begin{itemize}
    \item M (Markov) для входящего потока означает, что заявки поступают по экспоненциальному (пуассоновскому) закону с интенсивностью $\lambda$.
    \item M (Markov) для обслуживания говорит, что время обслуживания одной заявки также экспоненциально распределено с интенсивностью $\mu$.
    \item 1 --- количество каналов (серверов) обслуживания.
\end{itemize}

Коэффициент загрузки $\rho = \frac{\lambda}{\mu}$. Чтобы очередь не росла бесконечно, требуется $\rho < 1$

Среднее время ожидания в очереди $\frac{\rho}{\mu (1 - \rho)}$

Средняя длина очереди (количество заявок в очереди, не считая обслуживаемую): $L_q = \frac{\rho ^ 2}{1 - \rho}$

Среднее число заявок в системе (включая находящуюся на обслуживании): $L = \frac{\rho}{1 - \rho}$

Видно, что $L = L_q + \rho$

\textbf{Модель M/M/n}

Параметры:
\begin{enumerate}
    \item $\lambda$, $\mu$ как в модели M/M/1,
    \item $n$ --- число параллельных каналов (серверов),
    \item $\mu_n$ --- эффективная интенсивность обслуживания
\end{enumerate}

$\rho = \frac{\lambda}{n \mu} = \frac{\lambda}{\mu_n}$ --- <<эффективный>> коэффициент загрузки на каждый канал.

Вероятность системы быть пустой $P_0 = \left[ \sum \limits ^{n-1} _{k=0} \frac{(\lambda / \mu) ^ k}{k!} + \frac{(\lambda / \mu) ^ n}{n!} \frac{n\mu}{n\mu - \lambda} \right]$

Средняя длина очереди: $L_q = \frac{(\lambda / \mu) ^ n}{n! (1-\rho)} P_0$

Среднее время ожидания в очереди $\frac{\rho}{2 \mu (1 - \rho)}$

\textbf{Модель M/D/1}
\begin{itemize}
    \item M для поступления заявок остаётся таким же (пуассоновский поток),
    \item D (Deterministic) означает, что время обслуживания фиксировано (каждая заявка обслуживается ровно $1 / \mu$ времени),
    \item 1 --- один канал обслуживания.
\end{itemize}

Средняя длина очереди (количество заявок в очереди, не считая обслуживаемую): $L_q = \frac{\rho ^ 2}{2 (1 - \rho)}$

\textbf{Другие распределения}

В случаях выше использовалось распределение Пуассона, но могут быть выбраны другие, например, Эрланговское или Гиперэкспоненциальное.

\begin{itemize}
    \item Распределение Пуассона. Описывает количество событий $X$ (заявок, поступлений) за фиксированный промежуток времени при известной средней интенсивности $\lambda$. $$P(X=k) = \exp \left(-\lambda \frac{\lambda ^ k}{k!} \right)$$
    \item Эрланговское распределение. это частный случай гамма-распределения, где параметр формы $k$ (целое число) интерпретируется как количество последовательных стадий (фаз) экспоненциального процесса, каждая с интенсивностью $\lambda$. $$f(t) = \frac{\lambda^k t ^{k-1} e^{-\lambda t}}{(k-1)!}$$
    \item Гиперэкспоненциальное распределение представляет собой смесь нескольких экспоненциальных распределений, каждое с собственной интенсивностью $\lambda_i$, выбираемой с некоторой вероятностью $p_i$. При моделировании это соответствует сценарию, когда время обслуживания (или межприхода) может принадлежать «разным типам» экспоненциальных процессов. $$f(t) = \sum \limits ^m _{i=1} p_i \lambda _i e^{\lambda_i t}$$    
\end{itemize}

Роль цепей поставок в обеспечении конкурентного преимущества

\textbf{Основы управления цепями поставок}
Цепь поставок (supply chain) --- это совокупность всех процессов и участников, задействованных в создании и доставке продукта к конечному потребителю (от сырья до конечной реализации).Эффективная цепь поставок позволяет снизить затраты, повысить скорость и надежность поставок, а также гибко реагировать на изменения спроса.

\textbf{Конкурентные преимущества}
\begin{enumerate}
    \item Снижение совокупных издержек: продуманная логистика, оптимизация складских запасов и транспортных маршрутов.
    \item Увеличение ценности для клиента: оперативное реагирование на колебания спроса и индивидуализацию предложений.
    \item Интеграция и координация: совместное планирование с поставщиками и дистрибуторами, обмен информацией в режиме реального времени.
\end{enumerate}

\textbf{Стратегическое значение}
Современные компании все чаще рассматривают цепь поставок как ключевой фактор успеха --- она влияет не только на стоимость, но и на способность быстро выводить новую продукцию на рынок, удерживать клиентов и создавать инновационные модели взаимодействия. Инвестиции в инфраструктуру и технологии управления цепями поставок (например, системы управления складом, прогнозирование спроса, цифровые платформы) становятся долгосрочным источником конкурентных преимуществ.

\pagebreak

\subsection{Классическая структура цепи поставок, принципы её организации. Принципы управления запасами в цепи поставок (детерминированный и стохастический спрос). Различные типы договоров поставки для оптимизации работы всей цепи поставок}

\begin{itemize}
    \item \textbf{Участники цепи поставок:}
    \begin{itemize}
        \item Поставщики сырья и комплектующих;
        \item Производители (преобразование сырья в готовый продукт);
        \item Дистрибьюторы/оптовые продавцы (хранение, распределение);
        \item Розничные продавцы (магазины, онлайн-площадки);
        \item Конечные потребители.
    \end{itemize}
    \item \textbf{Принципы организации:}
    \begin{itemize}
        \item \textit{Интеграция и координация} — обмен информацией и совместное планирование (сокращение временных и денежных затрат).
        \item \textit{Логистическая концепция «точно в срок» (Just-in-time)} — сокращение запасов за счёт быстрой реакции и своевременных поставок.
        \item \textit{Баланс эффективности и гибкости} — оптимизация издержек при сохранении высокого уровня обслуживания.
        \item \textit{Системное видение} — все звенья рассматриваются как единое целое, а не обособленно.
    \end{itemize}
\end{itemize}

\textbf{Принципы управления запасами в цепи поставок}

\textbf{Детерминированный спрос}
\begin{itemize}
    \item \textbf{Особенности:}
    \begin{itemize}
        \item Спрос заранее известен (плановый, не меняется).
        \item Используются точные модели (например, модель экономичного размера заказа — EOQ).
    \end{itemize}
    \item \textbf{Основные параметры EOQ-модели:}
    \begin{itemize}
        \item $D$ — годовой объём спроса;
        \item $Q$ — размер партии заказа;
        \item $K$ — затраты на оформление одного заказа;
        \item $h$ — затраты на хранение единицы товара в год.
    \end{itemize}
    \[
        Q^* = \sqrt{\frac{2KD}{h}}, 
        \quad 
        \text{где } Q^* \text{ — оптимальный размер заказа.}
    \]
\end{itemize}

\textbf{Стохастический спрос}
\begin{itemize}
    \item \textbf{Особенности:}
    \begin{itemize}
        \item Спрос меняется во времени и носит случайный характер.
        \item Цель — найти оптимальный уровень страхового запаса для снижения вероятности дефицита.
    \end{itemize}
    \item \textbf{Подходы к управлению:}
    \begin{itemize}
        \item Модели с \textit{фиксированным интервалом пополнения} (periodic review).
        \item Модели с \textit{фиксированным размером заказа} (continuous review).
        \item \textit{Сервисный уровень} (принцип обслуживания): определяет, какую долю спроса необходимо удовлетворять своевременно.
    \end{itemize}
\end{itemize}

\textbf{Типы договоров поставки для оптимизации работы всей цепи}

\begin{itemize}
    \item \textbf{Контракт с оптовой ценой (Wholesale Price Contract):}
    \begin{itemize}
        \item Производитель назначает единую оптовую цену;
        \item Розничный продавец несёт риск остатков на складе (невозможность возврата);
        \item Может приводить к неэффективному распределению прибыли и не оптимизирует объём заказов с точки зрения всей цепи.
    \end{itemize}
    \item \textbf{Контракт с выкупом остатков (Buy-back Contract):}
    \begin{itemize}
        \item Производитель выкупает у ритейлера непроданные товары по заранее оговорённой цене;
        \item Снижает риск ритейлера и стимулирует его увеличивать заказ;
        \item Позволяет производителю и ритейлеру разделить риски и потенциальную прибыль.
    \end{itemize}
    \item \textbf{Контракт с разделением дохода (Revenue-Sharing Contract):}
    \begin{itemize}
        \item Производитель и ритейлер делят между собой доходы от продаж в определённой пропорции;
        \item Может стимулировать ритейлера вкладывать больше в маркетинг и поддержание спроса;
        \item Поощряет совместную оптимизацию, снижает общий уровень запасов.
    \end{itemize}
    \item \textbf{Контракт с гибкостью по объёму (Quantity Flexibility Contract):}
    \begin{itemize}
        \item Ритейлер может менять заказ (в пределах согласованных лимитов) в ответ на изменившийся спрос;
        \item Позволяет снизить риск дефицита или избытка на стороне ритейлера;
        \item Улучшает совокупную реакцию цепи поставок на колебания спроса.
    \end{itemize}
\end{itemize}

\pagebreak

\subsection{Финансовая модель технологического проекта. Цели финансового моделирования. Основные принципы и подходы к построению финансовой модели. Основные формы финансовой отчетности, их цели, задачи и структура. Взаимосвязь форм финансовой отчетности. Особенности отражения в отчетности расходов на R\&D и создания НМА.}

\textbf{Финансовая модель технологического проекта} --- это систематизированная модель, отражающая основные параметры и логику будущей деятельности компании в технологической сфере. В ней учитываются ожидания по выручке и рынку, особенности затрат, структура финансирования и перспективы роста стоимости бизнеса.

\textbf{Цели финансового моделирования:}
\begin{enumerate}
    \item \textit{Прогнозирование:} дать количественную оценку развития проекта (продажи, расходы, прибыль).
    \item \textit{Привлечение инвестиций:} обосновать потребность в финансировании и продемонстрировать инвестиционную привлекательность.
    \item \textit{Планирование и бюджетирование:} определить ключевые ресурсы и распределить их по этапам развития.
    \item \textit{Управление рисками:} выявить наиболее уязвимые места и проверить проект на устойчивость к неблагоприятным факторам.
\end{enumerate}

\textbf{Основные принципы и подходы к построению финансовой модели:}
\begin{enumerate}
    \item \textit{Прозрачность и детальность:} модель должна содержать понятные ссылки на исходные данные (допущения, прогнозы спроса, ставок и т.д.).
    \item \textit{Гибкость:} возможность быстро менять параметры (стоимость оборудования, темпы роста рынка), чтобы проводить сценарный и чувствительный анализ.
    \item \textit{Реалистичность:} использование правдоподобных предположений с учётом отраслевой и макроэкономической специфики.
    \item \textit{Взаимосвязанность блоков:} операционный блок (выручка, расходы), инвестиционный блок (капитальные затраты, R\&D), финансирование (кредиты, долевое участие), оценка окупаемости и распределение денежных потоков.
\end{enumerate}

\textbf{Основные формы финансовой отчетности:}
\begin{enumerate}
    \item \textit{Отчет о прибыли и убытках (P\&L)}
    
    Цель: показать доходы и расходы за период, определить финансовый результат (прибыль или убыток).
    
    Структура: выручка $\rightarrow$ себестоимость $\rightarrow$ валовая прибыль $\rightarrow$ операционные и прочие расходы $\rightarrow$ налоги $\rightarrow$ чистая прибыль.

    \item \textit{Баланс (Balance Sheet)}

    Цель: отразить финансовое состояние компании на определённую дату.
    
    Структура: активы (оборотные и внеоборотные), обязательства (краткосрочные и долгосрочные), собственный капитал.

    \item \textit{Отчет о движении денежных средств (Cash Flow Statement)}
    
    Цель: показать реальные потоки денежных средств за период, их источники и использование.
    
    Структура: операционная деятельность (CFO), инвестиционная деятельность (CFI), финансовая деятельность (CFF).
    
\end{enumerate}

\textbf{Взаимосвязь форм финансовой отчетности}

\begin{enumerate}
    \item \textit{Чистая прибыль} из P\&L переносится в раздел собственного капитала Баланса и влияет на нераспределенную прибыль.
    \item \textit{Движение денежных средств} зависит от чистой прибыли (корректировка на неденежные статьи) и изменений в статьях Баланса (например, дебиторская задолженность, запасы, кредиторская задолженность).
    \item \textit{Инвестиции в активы} отражаются в Балансе (например, увеличение нематериальных активов) и в разделе инвестиционной деятельности Отчета о движении денежных средств.
\end{enumerate}


\textbf{Особенности отражения в отчетности расходов на R\&D и создания НМА}
\textit{Расходы на R\&D} часто включают заработную плату, материалы, лабораторные расходы. В зависимости от стандартов (МСФО/РСБУ/GAAP) часть затрат может капитализироваться в нематериальные активы (НМА), если ожидается будущая экономическая выгода и соблюдены критерии признания актива.
\textit{Капитализированная часть R\&D} становится нематериальным активом в Балансе и амортизируется в течение срока полезного использования.
\textit{Некапитализированные затраты} отражаются как расходы периода в P\&L, что снижает чистую прибыль.



\subsection{Оценка инвестиционной привлекательности и эффективности технологического проекта. Расчет и экономический смысл коэффициентов эффективности проекта: NPV, IRR (MIRR), период окупаемости и дисконтированный период окупаемости, индекс прибыльности}

Оценка инвестиционной привлекательности технологического проекта основывается на сопоставлении будущих денежных потоков с первоначальными вложениями. 

\textbf{NPV (Net Present Value, Чистая приведенная стоимость)}

\textit{Что рассчитывает:} суммарную стоимость будущих денежных потоков, дисконтированных к текущему моменту, за вычетом первоначальных инвестиций.

\textit{Экономический смысл:} показывает абсолютный прирост капитала. Если NPV > 0, проект, как правило, считается эффективным и привлекательным для инвестирования.

\textit{Формула:}

\begin{equation}
    NVP = \sum \limits_{t=1} ^T \frac{CF_t}{(1+r)^t} - I_0,
\end{equation}
\noindent где
$CF_t$ --- денежный поток (приток или отток) в году $t$,
$r$ --- ставка дисконтирования (требуемая доходность),
$I_0$ --- первоначальные инвестиции (вложения),
$T$ --- горизонт планирования.


\textbf{IRR (Internal Rate of Return, Внутренняя норма доходности)}

\textit{Что рассчитывает:} ставку дисконтирования, при которой NPV проекта становится равной нулю. Сравнивается с требуемой доходностью (ставкой дисконтирования) инвестора: если IRR выше требуемой ставки --- проект привлекательнее.

\textit{Экономический смысл:} IRR сравнивают с требуемой доходностью. Если IRR выше требуемой ставки, проект выгоден для инвестирования.

\textit{Формула:}

$IRR$ --- это такая ставка $r_{IRR}$, при которой NPV проекта равна нулю:

\begin{equation}
    \sum \limits_{t=1} ^T \frac{CF_t}{(1+r_{IRR})^t} - I_0 = 0,
\end{equation}

\textbf{MIRR (Modified IRR)}
\textit{Что рассчитывает:} учитывает реинвестирование денежных потоков под заданную ставку и обычно отражает более реалистичную картину доходности проекта по сравнению с IRR.

\textit{Экономический смысл:} MIRR даёт более реалистичную оценку доходности, чем IRR, особенно если денежные потоки неоднородны во времени.

\textit{Формула:}

\begin{equation}
    MIRR = \sqrt[r]{\frac{\sum \limits ^T _{t=1} \left( \text{приведённый к концу проекта }CF_+ \right)}{\sum \limits ^T _{t=1} \left( \text{приведённый к концу проекта }CF_- \right)}} - 1
\end{equation}

В отличие от IRR, MIRR учитывает ставку реинвестирования положительных потоков и ставку финансирования (при необходимости) отрицательных потоков.


\textbf{Период окупаемости (Payback Period)}

\textit{Что рассчитывает:} за сколько лет суммарные поступления от проекта покроют изначальные вложения без учета дисконтирования.

\textit{Экономический смысл:} Показывает, за какое время проект «вернёт» вложенные средства. Не учитывает временную стоимость денег и притоки после окупаемости.

\textit{Формула (без дисконтирования):}
Суммируем годовые притоки $CF_t$ до тех пор, пока они не покроют первоначальные вложения $I_0$.
Период окупаемости --- это момент времени $T$ (в годах), когда $\sum \limits ^T _{t=1} CF_t \geq I_0$

\textbf{Дисконтированный период окупаемости (Discounted Payback Period)}

\textit{Что рассчитывает:} срок окупаемости с учетом дисконтирования будущих потоков.

\textit{Экономический смысл:} Даёт более точный срок возврата денег с учётом их временной стоимости.

\textit{Формула (c дисконтированием):}
Аналогично выше, но суммирование ведётся с учётом дисконтирования
\begin{equation}
    \sum \limits_{t=1} ^T \frac{CF_t}{(1+r)^t} \geq I_0
\end{equation}

$T$ --- минимальное целое число лет, за которое дисконтированные притоки покроют $I_0$.



\textbf{Индекс прибыльности (PI – Profitability Index)}

\textit{Что рассчитывает:} отношение суммы дисконтированных будущих денежных потоков к первоначальным вложениям.

\textit{Экономический смысл:} Если $PI > 1$, проект окупает вложения и приносит дополнительную ценность. Полезно при сравнении нескольких проектов с одинаковым объёмом капитала.

\textit{Формула:}
\begin{equation}
    PI = \frac{\sum \limits_{t=1} ^T \frac{CF_t}{(1+r)^t}}{I_0}
\end{equation}
\pagebreak


\subsection{Основные методы определения стоимости бизнеса. Понятие стоимости pre- и post-money. Принцип дисконтирования денежных потоков: понятие стоимости денег во времени, текущая и будущая стоимость. Определение ставки дисконтирования. Прогнозная и постпрогнозная (терминальная) стоимость. Модель Гордона для определения терминальной стоимости. Применимость доходного подхода для молодых инновационных компаний}

\textbf{Основные методы определения стоимости бизнеса}
\begin{enumerate}
    \item \textit{Затратный (имущественный) подход:} оценка стоимости на основе активов и обязательств компании.
    \item \textit{Сравнительный подход:} сопоставление с рыночной стоимостью аналогичных компаний (метод мультипликаторов).
    \item \textit{Доходный подход:} оценка на основе дисконтирования ожидаемых денежных потоков (DCF-анализ).
\end{enumerate}

\textbf{Pre-money и Post-money стоимость}
\begin{enumerate}
    \item \textit{Pre-money:} оценка стоимости компании \textit{до} привлечения нового капитала (инвестиций).
    \item \textit{Post-money:} оценка стоимости компании \textit{после} привлечения нового капитала (=\textit{Pre-money} + вложенные инвестиции).
\end{enumerate}

\textbf{Принцип дисконтирования денежных потоков}
\begin{itemize}
    \item \textit{Временная стоимость денег:} деньги сейчас ценнее тех же денег в будущем.
    \item \textit{Текущая (PV) и будущая (FV) стоимость:}
    \[
      \text{PV} = \frac{\text{FV}}{(1 + r)^n},
      \quad
      \text{FV} = \text{PV} \times (1 + r)^n,
    \]
    где $r$ --- ставка дисконтирования, $n$ --- число периодов.
\end{itemize}

\textbf{Определение ставки дисконтирования}
\begin{itemize}
    \item Зависит от уровня риска проекта, структуры капитала, отраслевых и рыночных факторов.
    \item На практике часто используют \textit{модель WACC} (средневзвешенная стоимость капитала) или \textit{CAPM} (модель оценки капитальных активов).
\end{itemize}

\textbf{Прогнозная и терминальная стоимость}
\begin{itemize}
    \item \textit{Прогнозная стоимость (Explícit Period):} сумма дисконтированных денежных потоков, рассчитанных по годам (обычно 3--5 лет).
    \item \textit{Терминальная (постпрогнозная) стоимость:} стоимость бизнеса после окончания прогнозного периода, учитывает стабильный темп роста.
\end{itemize}

\textbf{Модель Гордона (Gordon Growth Model)}
\begin{itemize}
    \item Используется для оценки \textit{терминальной стоимости} при постоянном темпе роста:
    \[
      \text{Terminal Value} = \frac{\text{FCF}_{n+1}}{(r - g)},
    \]
    где $\text{FCF}_{n+1}$~--- денежный поток на первый год после прогнозного периода, $r$~--- ставка дисконтирования, $g$~--- постоянный годовой темп роста.
\end{itemize}

\textbf{Доходный подход для молодых инновационных компаний}

Прогнозирование денежных потоков затруднено из-за риска и динамичных изменений.
\begin{itemize}
    \item \textit{Корректировка ставки дисконтирования:} выше из-за повышенного риска и возможности неудачи проекта.
    \item \textit{Комбинация подходов:} дополнение доходного подхода методами реальных опционов, привлечением сравнительных и аналитических оценок.
\end{itemize}
\pagebreak

\subsection{Оценка бизнеса методом венчурного капитала (в т.ч. в случае нескольких раундов инвестирования). Сравнительный подход (мультипликаторы): методы и порядок применения. Применимость сравнительного подхода для молодых инновационных компаний}

Метод венчурного капитала используется для оценки стартапов и молодых инновационных компаний. Этот подход основывается на прогнозировании будущей стоимости компании (exit value) и дисконтировании её до текущего момента с учётом высокого уровня риска.

Процесс оценки включает несколько этапов:
\begin{enumerate}
    \item Прогнозирование exit value компании, которое обычно основывается на рыночных мультипликаторах (например, EV/EBITDA или P/E), характерных для сравнимых компаний.
    \item Учет ожидаемой доли инвестора в будущем. Доля инвестора определяется как отношение вложенных средств к оценочной стоимости компании после инвестиций (post-money valuation).
    \item Применение дисконтирующего множителя, отражающего требуемую норму доходности венчурного инвестора, которая значительно выше среднего рыночного уровня из-за высокого риска.
\end{enumerate}

В случае нескольких раундов инвестирования процедура усложняется: для каждого раунда рассчитывается постинвестиционная стоимость, с учётом размывания доли инвестора на последующих этапах.

\textbf{Сравнительный подход (мультипликаторы): методы и порядок применения}

\begin{enumerate}
    \item Сравнительный подход основывается на анализе рыночной информации о сравнимых компаниях и использовании мультипликаторов для оценки. Основные этапы:
    \item Подбор сопоставимых компаний, находящихся в аналогичных рыночных условиях.
    \item Расчёт мультипликаторов на основе их финансовых показателей (например, EV/EBITDA, P/E, EV/Revenue).
    \item Применение выбранных мультипликаторов к соответствующим показателям оцениваемой компании для расчёта её стоимости.
\end{enumerate}

Этот метод особенно эффективен, если имеются данные о публичных компаниях или сделках, сравнимых с оцениваемым бизнесом.

\textbf{Применимость сравнительного подхода для молодых инновационных компаний}

Сравнительный подход имеет ограничения для оценки стартапов, так как финансовые показатели таких компаний часто нестабильны, а сопоставимые компании трудно определить. Однако подход может быть применим, если найдены компании с аналогичной бизнес-моделью, стадией развития или технологической направленностью. Для повышения точности используются отраслевые и региональные корректировки.

\pagebreak