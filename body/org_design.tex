\section{Организационный дизайн и стратегия фирмы}

\subsection{Три ключевых блока организации: организационный и стратегический дизайн, политическая система, корпоративная культура}

Организация как система включает в себя три ключевых блока: \textbf{организационный и стратегический дизайн}, \textbf{политическая система} и \textbf{корпоративная культура}. Эти элементы формируют целостность организации, определяя её устойчивость, способность к адаптации и достижения целей.

\begin{enumerate}
    \item Организационный и стратегический дизайн
        \begin{itemize}
            \item \textbf{Организационный дизайн}: 
            Определяет структуру организации, включая иерархию, распределение ролей и обязанностей, механизмы координации и коммуникации. Включает механистический (жёсткая структура, высокая формализация) и органический (гибкость, децентрализация) типы структур.
            \item \textbf{Стратегический дизайн}: 
            Ориентирован на формирование и реализацию стратегии фирмы, направленной на достижение конкурентных преимуществ. Включает в себя процесс стратегического планирования, анализа внешней и внутренней среды, постановку целей и определение ключевых ресурсов.
        \end{itemize}
    \item Политическая система
        \begin{itemize}
            \item Включает распределение власти, принятие решений, управление конфликтами и формирование коалиций.
            \item Отражает баланс интересов различных групп, такие как менеджмент, сотрудники, акционеры и внешние стейкхолдеры.
            \item Ключевые аспекты: 
            \begin{itemize}
                \item \textbf{Формальная власть}: официальные полномочия, закреплённые в структуре.
                \item \textbf{Неофициальная власть}: влияние через неформальные связи, авторитет, харизму и опыт.
            \end{itemize}
        \end{itemize}
    \item Корпоративная культура
        \begin{itemize}
            \item Определяет систему ценностей, норм, убеждений и поведения, разделяемых сотрудниками.
            \item Формирует основу для создания единства внутри организации, стимулирования лояльности и повышения эффективности.
            \item Ключевые элементы:
            \begin{itemize}
                \item \textbf{Артефакты}: видимые элементы культуры (дресс-код, офисное пространство).
                \item \textbf{Общие ценности}: ключевые убеждения, разделяемые большинством.
                \item \textbf{Основные допущения}: глубинные, часто неосознаваемые установки, формирующие восприятие реальности.
            \end{itemize}
            \item Типология культур: например, культура власти, роли, задачи и личности (по Хэнди).
        \end{itemize}
\end{enumerate}

Все три блока взаимозависимы:
\begin{itemize}
    \item Организационный дизайн определяет структуру и процессы, в рамках которых функционирует политическая система.
    \item Политическая система влияет на распределение ресурсов и поддержку стратегического дизайна.
    \item Корпоративная культура задаёт нормы поведения и восприятие изменений, поддерживая или ограничивая эффективность стратегии.
\end{itemize}

\pagebreak
\subsection{Управление организационными изменениями. Процесс управления изменениями. Модель анализа поля сил К. Левина. Стратегии проведения изменений: «сверху вниз» и «снизу вверх». Сопротивление организационным изменениям. Уровни сопротивления}

\textbf{Управление организационными изменениями}

Организационные изменения --- это процесс трансформации структуры, процессов, культуры или стратегии организации, направленный на адаптацию к внутренним и внешним условиям. Эффективное управление изменениями включает комплексный подход к их планированию, реализации и контролю.

\begin{enumerate}
    \item Процесс управления изменениями
        \begin{itemize}
            \item \textbf{Диагностика ситуации}: анализ текущего состояния организации, выявление проблем и возможностей.
            \item \textbf{Разработка плана изменений}: постановка целей, определение этапов и ресурсов.
            \item \textbf{Реализация изменений}: внедрение новых практик, технологий, структур.
            \item \textbf{Контроль и оценка}: мониторинг результатов, корректировка действий при необходимости.
        \end{itemize}
    \item Модель анализа поля сил К. Левина
        \begin{itemize}
            \item Основана на взаимодействии двух типов сил:
            \begin{itemize}
                \item \textbf{Движущие силы}: способствуют изменениям (технологические инновации, требования рынка).
                \item \textbf{Сдерживающие силы}: препятствуют изменениям (устойчивые привычки, страх перед новым).
            \end{itemize}
            \item Процесс изменений включает три стадии:
            \begin{itemize}
                \item \textbf{Размораживание}: осознание необходимости изменений, уменьшение влияния сдерживающих сил.
                \item \textbf{Изменение}: внедрение новых процессов, систем, поведения.
                \item \textbf{Замораживание}: закрепление изменений, создание новой стабильности.
            \end{itemize}
        \end{itemize}
    \item Стратегии проведения изменений
        \begin{itemize}
            \item \textbf{«Сверху вниз»}: инициатива исходит от руководства. Преимущества --- быстрая реализация, централизованный контроль; недостатки --- возможное сопротивление сотрудников.
            \item \textbf{«Снизу вверх»}: изменения инициируются на уровне сотрудников. Преимущества --- высокая вовлечённость персонала; недостатки --- долгий процесс внедрения, сложность координации.
        \end{itemize}
    \item Сопротивление организационным изменениям
        \begin{itemize}
            \item \textbf{Причины сопротивления}:
            \begin{itemize}
                \item Неуверенность в будущем, страх потери работы или статуса.
                \item Недостаток информации и понимания целей изменений.
                \item Приверженность привычным методам работы.
            \end{itemize}
            \item \textbf{Уровни сопротивления}:
            \begin{itemize}
                \item \textbf{Индивидуальный}: эмоциональная реакция, страх, стресс.
                \item \textbf{Групповой}: устоявшиеся нормы и ценности коллектива.
                \item \textbf{Организационный}: сопротивление из-за структурной инертности и недостатка ресурсов.
            \end{itemize}
        \end{itemize}
    \item Методы преодоления сопротивления
        \begin{itemize}
            \item Информирование и обучение сотрудников.
            \item Вовлечение персонала в процесс изменений.
            \item Поддержка и мотивация участников изменений.
            \item Постепенное внедрение и демонстрация «быстрых побед».
        \end{itemize}
\end{enumerate}

\pagebreak
\subsection{Понятие стратегии и стратегического управления. Виды бизнес-стратегий (диверсификация, интенсивный и интеграционный рост бизнеса)}




\pagebreak
\subsection{Модель анализа пяти конкурентных сил Майкла Портера. Методика проведения анализа и примеры использования при формировании корпоративной стратегии}




\pagebreak
\subsection{Отраслевая цепочка создания стоимости. Основные виды конкурентных стратегий. Понятие стратегии низких издержек, дифференциации и фокусирования}



\pagebreak
\subsection{Управление портфелем бизнеса. Инструменты портфельного анализа (матрицы BCG и McKinsey)}


