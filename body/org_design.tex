\section{Организационный дизайн и стратегия фирмы}

\subsection{Три ключевых блока организации: организационный и стратегический дизайн, политическая система, корпоративная культура}

Организация как система включает в себя три ключевых блока: \textbf{организационный и стратегический дизайн}, \textbf{политическая система} и \textbf{корпоративная культура}. Эти элементы формируют целостность организации, определяя её устойчивость, способность к адаптации и достижения целей.

\begin{enumerate}
    \item Организационный и стратегический дизайн
        \begin{itemize}
            \item \textbf{Организационный дизайн}: 
            Определяет структуру организации, включая иерархию, распределение ролей и обязанностей, механизмы координации и коммуникации. Включает механистический (жёсткая структура, высокая формализация) и органический (гибкость, децентрализация) типы структур.
            \item \textbf{Стратегический дизайн}: 
            Ориентирован на формирование и реализацию стратегии фирмы, направленной на достижение конкурентных преимуществ. Включает в себя процесс стратегического планирования, анализа внешней и внутренней среды, постановку целей и определение ключевых ресурсов.
        \end{itemize}
    \item Политическая система
        \begin{itemize}
            \item Включает распределение власти, принятие решений, управление конфликтами и формирование коалиций.
            \item Отражает баланс интересов различных групп, такие как менеджмент, сотрудники, акционеры и внешние стейкхолдеры.
            \item Ключевые аспекты: 
            \begin{itemize}
                \item \textbf{Формальная власть}: официальные полномочия, закреплённые в структуре.
                \item \textbf{Неофициальная власть}: влияние через неформальные связи, авторитет, харизму и опыт.
            \end{itemize}
        \end{itemize}
    \item Корпоративная культура
        \begin{itemize}
            \item Определяет систему ценностей, норм, убеждений и поведения, разделяемых сотрудниками.
            \item Формирует основу для создания единства внутри организации, стимулирования лояльности и повышения эффективности.
            \item Ключевые элементы:
            \begin{itemize}
                \item \textbf{Артефакты}: видимые элементы культуры (дресс-код, офисное пространство).
                \item \textbf{Общие ценности}: ключевые убеждения, разделяемые большинством.
                \item \textbf{Основные допущения}: глубинные, часто неосознаваемые установки, формирующие восприятие реальности.
            \end{itemize}
            \item Типология культур: например, культура власти, роли, задачи и личности (по Хэнди).
        \end{itemize}
\end{enumerate}

Все три блока взаимозависимы:
\begin{itemize}
    \item Организационный дизайн определяет структуру и процессы, в рамках которых функционирует политическая система.
    \item Политическая система влияет на распределение ресурсов и поддержку стратегического дизайна.
    \item Корпоративная культура задаёт нормы поведения и восприятие изменений, поддерживая или ограничивая эффективность стратегии.
\end{itemize}

\pagebreak
\subsection{Управление организационными изменениями. Процесс управления изменениями. Модель анализа поля сил К. Левина. Стратегии проведения изменений: «сверху вниз» и «снизу вверх». Сопротивление организационным изменениям. Уровни сопротивления}

\textbf{Управление организационными изменениями}

Организационные изменения --- это процесс трансформации структуры, процессов, культуры или стратегии организации, направленный на адаптацию к внутренним и внешним условиям. Эффективное управление изменениями включает комплексный подход к их планированию, реализации и контролю.

\begin{enumerate}
    \item Процесс управления изменениями
        \begin{itemize}
            \item \textbf{Диагностика ситуации}: анализ текущего состояния организации, выявление проблем и возможностей.
            \item \textbf{Разработка плана изменений}: постановка целей, определение этапов и ресурсов.
            \item \textbf{Реализация изменений}: внедрение новых практик, технологий, структур.
            \item \textbf{Контроль и оценка}: мониторинг результатов, корректировка действий при необходимости.
        \end{itemize}
    \item Модель анализа поля сил К. Левина
        \begin{itemize}
            \item Основана на взаимодействии двух типов сил:
            \begin{itemize}
                \item \textbf{Движущие силы}: способствуют изменениям (технологические инновации, требования рынка).
                \item \textbf{Сдерживающие силы}: препятствуют изменениям (устойчивые привычки, страх перед новым).
            \end{itemize}
            \item Процесс изменений включает три стадии:
            \begin{itemize}
                \item \textbf{Размораживание}: осознание необходимости изменений, уменьшение влияния сдерживающих сил.
                \item \textbf{Изменение}: внедрение новых процессов, систем, поведения.
                \item \textbf{Замораживание}: закрепление изменений, создание новой стабильности.
            \end{itemize}
        \end{itemize}
    \item Стратегии проведения изменений
        \begin{itemize}
            \item \textbf{«Сверху вниз»}: инициатива исходит от руководства. Преимущества --- быстрая реализация, централизованный контроль; недостатки --- возможное сопротивление сотрудников.
            \item \textbf{«Снизу вверх»}: изменения инициируются на уровне сотрудников. Преимущества --- высокая вовлечённость персонала; недостатки --- долгий процесс внедрения, сложность координации.
        \end{itemize}
    \item Сопротивление организационным изменениям
        \begin{itemize}
            \item \textbf{Причины сопротивления}:
            \begin{itemize}
                \item Неуверенность в будущем, страх потери работы или статуса.
                \item Недостаток информации и понимания целей изменений.
                \item Приверженность привычным методам работы.
            \end{itemize}
            \item \textbf{Уровни сопротивления}:
            \begin{itemize}
                \item \textbf{Индивидуальный}: эмоциональная реакция, страх, стресс.
                \item \textbf{Групповой}: устоявшиеся нормы и ценности коллектива.
                \item \textbf{Организационный}: сопротивление из-за структурной инертности и недостатка ресурсов.
            \end{itemize}
        \end{itemize}
    \item Методы преодоления сопротивления
        \begin{itemize}
            \item Информирование и обучение сотрудников.
            \item Вовлечение персонала в процесс изменений.
            \item Поддержка и мотивация участников изменений.
            \item Постепенное внедрение и демонстрация «быстрых побед».
        \end{itemize}
\end{enumerate}

\pagebreak
\subsection{Понятие стратегии и стратегического управления. Виды бизнес-стратегий (диверсификация, интенсивный и интеграционный рост бизнеса)}

\textbf{Стратегия} --- это долгосрочный план действий, направленный на достижение целей организации, её конкурентных преимуществ и адаптацию к изменениям во внешней среде. Стратегия определяет основные направления распределения ресурсов и выбор путей развития компании.

Ключевые аспекты стратегии:
\begin{itemize}
    \item Визия будущего состояния компании.
    \item Определение конкурентных преимуществ.
    \item Баланс между внутренними ресурсами и внешними возможностями.
\end{itemize}

\textbf{Понятие стратегического управления}
\textbf{Стратегическое управление} --- это процесс разработки, реализации и контроля за выполнением стратегии организации. Включает:
\begin{itemize}
    \item Анализ внешней и внутренней среды (SWOT-анализ, PESTEL-анализ).
    \item Постановку стратегических целей.
    \item Формулирование, внедрение и оценку стратегии.
\end{itemize}

\textbf{Виды бизнес-стратегий}
Бизнес-стратегии направлены на рост и развитие компании, разделяются на следующие виды:

\textit{Диверсификация}
\begin{itemize}
    \item \textbf{Определение}: Расширение продуктового портфеля или выход на новые рынки.
    \item \textbf{Типы диверсификации}:
    \begin{itemize}
        \item \textbf{Связанная диверсификация}: запуск продуктов или услуг, связанных с текущим бизнесом (например, расширение линейки).
        \item \textbf{Несвязанная диверсификация}: выход в новые, несвязанные отрасли (например, крупная корпорация начинает работать в другой сфере).
    \end{itemize}
    \item Преимущества: снижение рисков, увеличение рыночных возможностей.
    \item Недостатки: сложность управления разнородным бизнесом.
\end{itemize}

\textit{Интенсивный рост}
\begin{itemize}
    \item \textbf{Определение}: Увеличение объёма продаж текущих продуктов или услуг в рамках существующего бизнеса.
    \item \textbf{Стратегии интенсивного роста}:
    \begin{itemize}
        \item \textbf{Проникновение на рынок}: увеличение доли на существующих рынках (например, активные маркетинговые кампании).
        \item \textbf{Развитие рынка}: выход на новые рынки с существующими продуктами.
        \item \textbf{Развитие продукта}: улучшение или модификация текущих продуктов.
    \end{itemize}
    \item Преимущества: фокус на основном бизнесе.
    \item Недостатки: ограниченность ресурсов и возможности насыщения рынка.
\end{itemize}

\textit{Интеграционный рост}
\begin{itemize}
    \item \textbf{Определение}: Расширение бизнеса за счёт контроля над звеньями цепочки поставок.
    \item \textbf{Типы интеграции}:
    \begin{itemize}
        \item \textbf{Вертикальная интеграция}: контроль над поставщиками (прямой интеграцией) или дистрибьюторами (обратной интеграцией).
        \item \textbf{Горизонтальная интеграция}: поглощение или слияние с конкурентами.
    \end{itemize}
    \item Преимущества: снижение издержек, усиление рыночных позиций.
    \item Недостатки: высокие затраты на реализацию.
\end{itemize}



\pagebreak
\subsection{Модель анализа пяти конкурентных сил Майкла Портера. Методика проведения анализа и примеры использования при формировании корпоративной стратегии}

\section*{Модель анализа пяти конкурентных сил Майкла Портера}

Модель пяти сил Майкла Портера используется для анализа конкурентной среды в отрасли и разработки корпоративной стратегии. Она позволяет оценить уровень конкуренции и определить привлекательность рынка.

\begin{enumerate}
    \item \textbf{Угрозы со стороны новых конкурентов}
    \begin{itemize}
        \item Зависит от барьеров входа: капиталоёмкость, экономика масштаба, доступ к каналам сбыта, правовые ограничения.
        \item Высокие барьеры снижают угрозу, низкие --- усиливают.
    \end{itemize}

    \item \textbf{Угрозы со стороны товаров-заменителей (субститутов)}
    \begin{itemize}
        \item Зависит от доступности и привлекательности альтернативных товаров или услуг.
        \item Высокая угроза, если субституты дешевле, качественнее или удовлетворяют потребности лучше.
    \end{itemize}

    \item \textbf{Сила поставщиков}
    \begin{itemize}
        \item Зависит от концентрации поставщиков, важности их продукции, возможности интеграции поставщиков.
        \item Высокая сила поставщиков повышает издержки компании.
    \end{itemize}

    \item \textbf{Сила покупателей}
    \begin{itemize}
        \item Зависит от концентрации покупателей, доступности альтернативных поставщиков, чувствительности к цене.
        \item Высокая сила покупателей может снижать прибыльность компании.
    \end{itemize}

    \item \textbf{Уровень конкуренции среди существующих игроков}
    \begin{itemize}
        \item Зависит от количества конкурентов, темпов роста рынка, барьеров выхода, дифференциации продукта.
        \item Высокая конкуренция снижает прибыльность всей отрасли.
    \end{itemize}
\end{enumerate}

\textbf{Методика проведения анализа}
\begin{enumerate}
    \item \textit{Сбор данных об отрасли}: изучение рыночных характеристик, ключевых игроков, технологических трендов.
    \item \textit{Оценка каждой силы}: выявление её влияния на отрасль, использование качественных и количественных данных.
    \item \textit{Идентификация слабых и сильных сторон отрасли}: определение возможностей для снижения угроз и усиления позиций компании.
\end{enumerate}

\textbf{Примеры использования при формировании корпоративной стратегии}
\begin{itemize}
    \item \textit{Угрозы новых конкурентов}: 
    Реализация стратегии дифференциации продукта, чтобы усложнить вход новым игрокам.
    \item \textit{Товары-заменители}: 
    Инвестиции в инновации и улучшение качества продукта для повышения лояльности клиентов.
    \item \textit{Сила поставщиков}: 
    Стратегия вертикальной интеграции, чтобы взять под контроль ключевые звенья цепочки поставок.
    \item \textit{Сила покупателей}: 
    Увеличение ценности продукта за счёт персонализации или дополнительных услуг.
    \item \textit{Конкуренция внутри отрасли}: 
    Слияния и поглощения для сокращения количества игроков и усиления рыночной доли.
\end{itemize}

\textbf{Преимущества модели}
\begin{itemize}
    \item Обеспечивает системное понимание отраслевой динамики.
    \item Помогает выявить ключевые факторы для улучшения конкурентной позиции.
\end{itemize}

\textbf{Ограничения модели}
\begin{itemize}
    \item Не учитывает динамические изменения среды (например, инновации, глобализацию).
    \item Применима преимущественно к зрелым отраслям с устойчивыми структурами.
\end{itemize}


\pagebreak
\subsection{Отраслевая цепочка создания стоимости. Основные виды конкурентных стратегий. Понятие стратегии низких издержек, дифференциации и фокусирования}

\textbf{Отраслевая цепочка создания стоимости}

\textbf{Цепочка создания стоимости} (value chain) --- это последовательность этапов, через которые проходит продукт или услуга, начиная с разработки и производства и заканчивая доставкой клиенту. Цель анализа цепочки --- выявить ключевые элементы, создающие ценность для клиента, и определить области для повышения эффективности.

\begin{itemize}
    \item \textit{Основные этапы цепочки создания стоимости}:
    \begin{itemize}
        \item Исследование и разработка (R\&D).
        \item Производство.
        \item Логистика и дистрибуция.
        \item Маркетинг и продажи.
        \item Послепродажное обслуживание.
    \end{itemize}
    \item \textit{Ценность для клиента} создаётся за счёт оптимизации процессов и снижения издержек или улучшения качества продукта.
\end{itemize}

\textbf{Основные виды конкурентных стратегий}
По модели Майкла Портера существуют три базовые конкурентные стратегии:
\begin{enumerate}
    \item \textit{Стратегия низких издержек}.
    \item \textit{Стратегия дифференциации}.
    \item \textit{Стратегия фокусирования}.
\end{enumerate}

\textbf{Стратегия низких издержек}
\begin{itemize}
    \item \textit{Суть}: компания стремится стать лидером по издержкам, предлагая конкурентоспособные цены.
    \item \textit{Основные механизмы}: использование экономики масштаба, автоматизация, стандартизация, оптимизация цепочки поставок.
    \item \textit{Пример}: ритейлеры, такие как Walmart, которые снижают издержки и предлагают низкие цены.
    \item \textit{Риски}: снижение качества, невозможность дальнейшего сокращения издержек.
\end{itemize}

\textbf{Стратегия дифференциации}
\begin{itemize}
    \item \textit{Суть}: компания предлагает уникальный продукт или услугу, за которые клиенты готовы платить премиальную цену.
    \item \textit{Основные механизмы}: инновации, высокое качество, уникальный дизайн, брендинг, исключительный клиентский сервис.
    \item \textit{Пример}: компании Apple, Tesla, которые выделяются уникальными продуктами.
    \item \textit{Риски}: высокие издержки на инновации и разработку, риск копирования со стороны конкурентов.
\end{itemize}

\textbf{Стратегия фокусирования}
\begin{itemize}
    \item \textit{Суть}: компания концентрируется на узком рыночном сегменте, удовлетворяя его потребности лучше, чем конкуренты.
    \item \textit{Подвиды}:
    \begin{itemize}
        \item \textit{Фокус на низких издержках}: предложение продуктов по минимальным ценам в рамках нишевого рынка.
        \item \textit{Фокус на дифференциации}: уникальные продукты или услуги для узкого сегмента.
    \end{itemize}
    \item \textit{Пример}: производители люксовых товаров, таких как Rolex, или компании, работающие в специализированных нишах.
    \item \textit{Риски}: узость сегмента может ограничивать возможности роста.
\end{itemize}

\textbf{Применение стратегий в практике}
\begin{itemize}
    \item Компании могут комбинировать стратегии, однако риск «застрять посередине» (неэффективная реализация низких издержек или дифференциации) может снизить их конкурентоспособность.
    \item Выбор стратегии зависит от анализа отраслевой цепочки создания стоимости и конкурентной среды.
\end{itemize}

\pagebreak
\subsection{Управление портфелем бизнеса. Инструменты портфельного анализа (матрицы BCG и McKinsey)}

\textbf{Управление портфелем бизнеса} --- это процесс распределения ресурсов между бизнес-направлениями компании с целью достижения стратегических целей, повышения прибыльности и устойчивости. Основная цель --- оптимизация баланса между риском и доходностью.

Задачи портфельного управления:
\begin{itemize}
    \item Оценка текущей позиции бизнес-направлений.
    \item Определение стратегических приоритетов.
    \item Распределение инвестиций между направлениями.
\end{itemize}

\textbf{Матрица BCG (Boston Consulting Group)}

Матрица BCG --- инструмент для анализа бизнес-портфеля компании, основанный на двух критериях:
\begin{itemize}
    \item \textbf{Темп роста рынка}.
    \item \textbf{Доля рынка компании относительно конкурентов}.
\end{itemize}

Четыре категории матрицы:
\begin{itemize}
    \item \textbf{<<Звёзды>>}: Высокая доля рынка и высокий темп роста. Требуют значительных инвестиций для поддержания роста. Потенциально превращаются в «дойных коров».
    \item \textbf{<<Дойные коровы>>}: Высокая доля рынка при низком темпе роста. Генерируют стабильные доходы, могут финансировать другие направления.
    \item \textbf{<<Вопросительные знаки>>}: Низкая доля рынка, но высокий темп роста. Требуют значительных вложений, чтобы стать <<звёздами>>, либо выход из бизнеса при недостаточной перспективе.
    \item \textbf{<<Собаки>>}: Низкая доля рынка и низкий темп роста. Рекомендуется минимизация вложений или выход из данного сегмента.
\end{itemize}

\textbf{Применение матрицы BCG}:
\begin{itemize}
    \item Упрощённая оценка текущего состояния бизнеса.
    \item Выявление направлений для инвестиций и дезинвестиций.
\end{itemize}

\textbf{Матрица McKinsey (GE Matrix)}
Матрица McKinsey позволяет анализировать бизнес-портфель с учётом большего количества факторов. Критерии оценки:
\begin{itemize}
    \item \textbf{Привлекательность отрасли}: темпы роста, уровень конкуренции, рентабельность.
    \item \textbf{Конкурентная позиция}: доля рынка, доступ к ресурсам, инновационные возможности.
\end{itemize}

Структура матрицы:
\begin{itemize}
    \item 9 ячеек, формируемых на пересечении осей «привлекательность отрасли» (низкая, средняя, высокая) и «конкурентная позиция» (слабая, средняя, сильная).
    \item Стратегические рекомендации зависят от положения бизнеса в матрице:
    \begin{itemize}
        \item \textbf{Инвестировать}: высокая привлекательность отрасли и сильная конкурентная позиция.
        \item \textbf{Поддерживать}: средние показатели.
        \item \textbf{Дезинвестировать}: низкая привлекательность и слабая позиция.
    \end{itemize}
\end{itemize}

\textit{Преимущества матрицы McKinsey}:
\begin{itemize}
    \item Учитывает многомерность факторов.
    \item Гибкость анализа в различных отраслях.
\end{itemize}

\textit{Ограничения}:
\begin{itemize}
    \item Сложность и субъективность оценки параметров.
    \item Необходимость значительного объёма данных.
\end{itemize}

\textbf{Применение портфельного анализа}
\begin{itemize}
    \item Распределение инвестиций для максимизации прибыли.
    \item Оптимизация структуры бизнеса в соответствии с корпоративной стратегией.
    \item Принятие решений о выводе из нерентабельных сегментов.
\end{itemize}
